\documentclass{book}
\usepackage[spanish]{babel} % Agregamos la ortografía española
\usepackage[utf8]{inputenc} % Agregamos las tildes
\usepackage{amsmath} % Sin este no podría usar ciertos entornos como el de ecuación
\usepackage{amssymb} % Sin este no puedo usar algunos símbolos matemáticos



\begin{document}
	\tableofcontents
	\chapter{Introducción}
	En este libro vamos a tratar el concepto de los libros en latex
	\section{Pequeño repaso}
	Vemos que podemos hacer básicamente lo mismo que con un artículo
	\chapter{El modelo matemático de flexión de una viga}
	Al igual que podemos crear un índice para el libro, también podemos crear un índice de figuras y otro de tablas.

Usando el teorema dado en el libro \cite{articulo1} \cite{articulo2}

% \begin{thebibliography}{99}
%\bibitem{libro1} Pérez ortega, Cálculo I, Ed. alhambra 2008
% \end{thebibliography}

\bibliography{bibliografia}
\bibliographystyle{plain}
	
	
\end{document}